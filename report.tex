\documentclass[12pt, letterpaper]{article}
\usepackage[margin=1in]{geometry}
\usepackage{mathptmx}
\usepackage{amsmath}
\usepackage{graphicx}
\usepackage{booktabs}
\usepackage[font={small,it}, justification=centering, labelfont=bf]{caption}
\usepackage{float}
\usepackage[skip=4pt]{parskip}
\usepackage{hyperref}
\hypersetup{colorlinks=true, linkcolor=blue, urlcolor=blue}
\usepackage{titlesec}
\titlespacing*{\section}{0pt}{8pt}{4pt}
\titlespacing*{\subsection}{0pt}{6pt}{2pt}

% Roman numeral section counter
\renewcommand{\thesection}{\Roman{section}}
\titleformat{\section}{\normalfont\bfseries\large}{\thesection.}{0.5em}{}

% Simple arabic numbering for subsections (1. 2. 3. not III.1)
\renewcommand{\thesubsection}{\arabic{subsection}}
\titleformat{\subsection}{\normalfont\bfseries}{\thesubsection.}{0.5em}{}

\begin{document}

\begin{center}
  {\large \textbf{Report for CDA 3201L Lab \#3}}\\[2pt]
  {\large \textbf{Ring Oscillator with 3 NAND}}\\[8pt]
  \begin{tabular}{ll}
    \textbf{Name:} Tuan Khang Phan & \textbf{UID:} 030382645 \\
    \textbf{Name:} Huu Phat Nguyen & \textbf{UID:} U46082380 \\
  \end{tabular}
\end{center}

\section{Introduction}
The goal of this lab is to learn how to use the oscilloscope tool in the Analog Discovery II device. This helps us understand how to analyze and debug signals in a digital circuit. For this purpose, we built and tested a ring oscillator circuit using three NAND gates from a 74LS00 IC. We then measured the period and frequency of the output signal using the oscilloscope and compared the results with theoretical values calculated from the datasheet.

\section{Methods and Materials}
To complete the circuit, we needed the following equipment:
\begin{itemize}
  \item SN74LS00N IC
  \item Breadboard
  \item Jumper wires
  \item Alligator clips
  \item Analog Discovery II (oscilloscope)
  \item WaveForms software
  \item 5\,V DC power supply
\end{itemize}

Three NAND gates from a single SN74LS00N IC were each configured as inverters by connecting both inputs together. The output of the third gate was fed back to the input of the first gate to create a closed loop. One input of the first gate was used as the enable (EN) line and was connected to HIGH to allow oscillation.

We connected Channel 1 of the Analog Discovery II oscilloscope to the output node Z of the circuit. In WaveForms, we set the time base to 50\,ns/div and configured the trigger to rising edge on Channel 1 at 0\,V. We used X cursors placed on consecutive rising edges to measure the period directly. The Measurements panel was also used to cross-check the frequency reading.

\section{Results}

\subsection{Period Measured with Cursor}
Using X cursors in Delta mode placed on consecutive rising edges, the period was read from the delta X value:

$\Delta X = 201.4$\,ns

This corresponds to a frequency of $1 / \Delta X = 1 / (201.4 \times 10^{-9}) \approx 4.965$\,MHz.

\subsection{Computed Frequency}
\[
f = \frac{1}{T} = \frac{1}{201.4 \times 10^{-9}} \approx 4.965\ \text{MHz}
\]

\subsection{Measured Frequency}
The Measurements panel in WaveForms reported a frequency of 4.9983\,MHz with a period of 200.07\,ns. This is consistent with the cursor measurement within about 0.7\%.

\subsection{Oscilloscope Screenshot}

\begin{figure}[H]
\centering
\includegraphics[width=0.95\linewidth]{dashboard.jpg}
\caption{WaveForms oscilloscope display showing the ring oscillator output waveform with X cursor period measurements.}
\end{figure}
\subsection{Physical Circuit}

\begin{figure}[H]
\centering
\includegraphics[width=0.55\linewidth]{circuit.jpg}
\caption{Breadboard implementation of the ring oscillator using SN74LS00N.}
\end{figure}

\section{Prelab Question Answers}

\textbf{PQ-1.} From the SN74LS00 datasheet switching characteristics:

Typical propagation delay: $(3 + 10 + 15 + 4) / 4 = 8$\,ns

Maximum propagation delay: 15\,ns

\textbf{PQ-2.} Continuing from the exercise where the output of U2 is HIGH ($Z = 1$, 30\,ns elapsed using $t_p = 10$\,ns per gate):

The HIGH output from U2 is fed into the input of U0. U0 then produces a LOW output, which is sent to U1. U1 outputs HIGH, and this signal is fed back into U2, resulting in a LOW final output ($Z = 0$). This second half of the cycle takes another $3 \times 10 = 30$\,ns.

In total, the time that has passed since $Z = 0$ is $30 + 30 = 60$\,ns, which is one full period of oscillation.

\textbf{PQ-3.} When EN $= 1$, each NAND gate operates normally. In this configuration, each gate effectively functions as an inverter. Since there are three inverters connected in a loop (an odd number of inversions), the signal continuously inverts as it circulates, causing oscillation.

When EN $= 0$, the output of each NAND gate is forced to 1, regardless of the feedback signal. Because all outputs remain HIGH, no inversion occurs, and therefore oscillation does not take place. The oscillator is disabled.

\textbf{PQ-4.} The oscillation period depends on the propagation delay of the gates:
\[
T = 2 \times G \times t_{pd}
\]
For $G = 3$ gates with typical $t_{pd} = 8$\,ns:
\[
T = 2 \times 3 \times 8 = 48\ \text{ns}, \quad f = \frac{1}{48 \times 10^{-9}} \approx 20.83\ \text{MHz}
\]
For maximum $t_{pd} = 15$\,ns:
\[
T = 2 \times 3 \times 15 = 90\ \text{ns}, \quad f = \frac{1}{90 \times 10^{-9}} \approx 11.1\ \text{MHz}
\]

\section{Discussion}
The measured period of about 200\,ns (4.9983\,MHz) is much longer than the predicted typical period of 48\,ns (20.83\,MHz). If we back-calculate the per-gate delay from the scope measurement:
\[
t_{pd,\text{meas}} = \frac{T}{2G} = \frac{200.07}{6} \approx 33.3\ \text{ns}
\]
This is more than double the datasheet maximum of 15\,ns. The main reason for this difference is the extra capacitance from the breadboard and jumper wires. The datasheet test conditions use $C_L = 15$\,pF, but breadboard wiring can add much more capacitance to each node, which slows down the transitions. This can be seen in the waveform where the edges are rounded instead of sharp, and the output voltage only swings from about 0\,V to 1.4\,V instead of the full 0 to 5\,V range. The cursor measurement (201.4\,ns) and the automatic scope measurement (200.07\,ns) agreed within about 0.7\%. The enable mechanism worked correctly: setting EN to LOW stopped the oscillation, and setting EN to HIGH resumed it, which matches the expected behavior from PQ-3.

\section{Conclusion}
In this lab, we successfully built and tested a ring oscillator using three NAND gates from the 74LS00 IC. We used the Analog Discovery II oscilloscope to measure the period and frequency of the output signal. The measured frequency of about 4.97\,MHz was lower than the theoretical prediction of 20.83\,MHz. This difference was mainly caused by the extra capacitance from the breadboard wiring, which increased the effective propagation delay per gate to about 33.3\,ns. The cursor measurement and the automatic scope measurement agreed within 0.7\%, and the enable function worked as expected. This experiment helped us understand how propagation delay determines oscillation frequency and why circuit layout affects the performance of digital circuits.

\end{document}
